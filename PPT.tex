\documentclass{beamer}
\usepackage[table]{xcolor}
\usepackage{array,multirow}
\usepackage{tikz}
\usetikzlibrary{decorations.pathreplacing}
\usepackage{ctex, hyperref}
\usepackage[T1]{fontenc}

% other packages
\usepackage{latexsym,amsmath,xcolor,multicol,booktabs,calligra}
\usepackage{graphicx,pstricks,listings,stackengine}


\title{$\texttt{Python}$ 第三课}
\subtitle{列表 $\texttt{(list)}$}
\institute{香港中文大学(深圳) \ 数据科学学院}
\date{2025 年 4 月 12 日}
\author{蒋一歌\ 孙超逸}
\usepackage{CUHKSZ}

% defs
\def\cmd#1{\texttt{\color{red}\footnotesize $\backslash$#1}}
\def\env#1{\texttt{\color{blue}\footnotesize #1}}
\xdefinecolor{cuhksz1}{rgb}{0.457,0.0585,0.42578}      %RGB(117,15,109)
\xdefinecolor{cuhksz2}{rgb}{0.86328,0.63671875,0}      %RGB(221,163,0)
\xdefinecolor{cuhksz3}{rgb}{0.953125,0.87109,0.6875}   %RGB(244,223,176)


\definecolor{deepred}{rgb}{0.6,0,0}

\lstset{
    basicstyle=\ttfamily\small,
    keywordstyle=\bfseries\color{cuhksz1},
    emphstyle=\ttfamily\color{deepred},    % Custom highlighting style
    stringstyle=\color{cuhksz2},
    numbers=left,
    numberstyle=\small\color{cuhksz3},
    rulesepcolor=\color{red!20!green!20!blue!20},
    frame=shadowbox,
}


\begin{document}

\kaishu
\begin{frame}
    \titlepage
    \begin{figure}[htpb]
        \begin{center}
            \includegraphics[width=0.3\linewidth]{pic/CUHKSZ-Logo.pdf}
        \end{center}
    \end{figure}
\end{frame}



\section{列表的创建}

% 方式1:直接创建
\begin{frame}[fragile]{创建方式 $1$-直接定义}
\begin{exampleblock}{使用方括号直接创建}
\begin{lstlisting}[language=Python]
list_a = ["皮卡丘","哆啦A梦","孙悟空"]
list_b = [98,87,92,85] 
list_c = ["苹果",3,True]
\end{lstlisting}
\end{exampleblock}

\end{frame}

% 方式2:类型转换 
\begin{frame}[fragile]{创建方式 $2$-类型转换}
\begin{exampleblock}{$\texttt{list ()}$ 转换}
\begin{lstlisting}[language=Python]
list_a = list("abc") # ['a', 'b', 'c']
list_b = list((4,5,6)) # [4, 5, 6]
\end{lstlisting}
\end{exampleblock}

\end{frame}

% 方式3:range
\begin{frame}[fragile]{创建方式 $3$- $\texttt{range}$ 生成}
\begin{exampleblock}{$\texttt{range (start,end[,step])}$}
\begin{lstlisting}[language=Python]
list_a = list(range(3)) # [0, 1, 2]
list_b = list(range(3,7)) # [3, 4, 5, 6]
list_c = list(range(9,3,-2)) # [9, 7, 5]
\end{lstlisting}
\end{exampleblock}

\end{frame}


\section{列表索引与切片}
\begin{frame}[fragile]{访问列表元素}
\begin{block}{基础索引}
\begin{lstlisting}[language=Python]
stu = ["张三","李四","王五","赵六"]

# 获取单个元素
print(stu[0])  # 输出: "张三" 
print(stu[-1]) # 输出: "赵六" (倒数第一个)
\end{lstlisting}
\begin{itemize}
\item 索引从\colorbox{cuhksz3}{0}开始计数
\item 负数索引表示\colorbox{cuhksz3}{从后向前}访问
\item 越界访问会引发\colorbox{cuhksz3}{IndexError}
\end{itemize}
\end{block}

\end{frame}

\begin{frame}[fragile]{神奇的切片操作}
\begin{block}{切片语法}
\centering
\texttt{列表[开始:结束:步长]}
\end{block}

\begin{exampleblock}{实际应用}
\begin{lstlisting}[language=Python]
num = [0,1,2,3,4,5,6,7,8,9]

print(num[2:5])   # [2,3,4] (左闭右开)
print(num[:3])    # [0,1,2] (省略开始)
print(num[6:])    # [6,7,8,9] (省略结束)
print(num[::2])   # [0,2,4,6,8] (步长为2)
print(num[::-1])  # 逆序 [9,8,...,0]
\end{lstlisting}
\end{exampleblock}
\end{frame}

\begin{frame}{神奇的切片操作}
\input{silce.tex}
\begin{itemize}
\item \textcolor{blue}{蓝色}:$[2:5] \to [2,3,4]$
\item \textcolor{red}{红色}:$[1:8:2] \to [1,3,5,7]$
\item \textcolor{green}{绿色}:$[::-1]  \to \text{逆序}$
\end{itemize}
\begin{alertblock}{注意}
\begin{itemize}
\item 产生\colorbox{cuhksz3}{新列表}不影响原数据
\item 结束索引\colorbox{cuhksz3}{不包含}在结果中
\item 步长可为负数实现\colorbox{cuhksz3}{逆序}
\end{itemize}
\end{alertblock}
\end{frame}

\begin{frame}[fragile]{实战练习}
\begin{block}{基础练习}
\begin{lstlisting}[language=Python]
lst = ['A','B','C','D','E','F']

# 1. 获取前3个字母
print(lst[_____])

# 2. 获取最后2个字母
print(lst[_____])

# 3. 获取B到D的子列表
print(lst[_____])
\end{lstlisting}
\end{block}
\end{frame}

\begin{frame}[fragile]{实战练习}
\begin{block}{进阶挑战}
\begin{lstlisting}[language=Python]
num = list(range(20))

# 1. 获取所有偶数索引的元素
print(num[_____])

# 2. 创建逆序列表
inv = num[_____]

# 3. 获取第5到第15的元素,步长3
print(num[_____])
\end{lstlisting}
\end{block}
\end{frame}

\begin{frame}[fragile]{参考答案}
\begin{block}{基础练习答案}
\begin{lstlisting}[language=Python]
print(lst[:3])    # ['A','B','C']
print(lst[-2:])   # ['E','F'] 
print(lst[1:4])   # ['B','C','D']
\end{lstlisting}
\end{block}

\begin{block}{进阶挑战答案}
\begin{lstlisting}[language=Python]
print(num[::2])    # 偶数索引
inv = num[::-1]
print(num[5:16:3]) # 步长3
\end{lstlisting}
\end{block}
\end{frame}


\section{列表操作符}
\begin{frame}[fragile]{列表的操作符}
\begin{block}{列表的运算符重载}
\begin{tabular}{|l|l|p{2cm}|}
\hline
\textcolor{cuhksz1}{表达式} & \textcolor{cuhksz1}{结果} & \textcolor{cuhksz1}{描述} \\ \hline
\lstinline|[1,2,3] + [4,5]| & \lstinline|[1,2,3,4,5]| & 列表拼接\\ \hline
\lstinline|['A','B'] * 3| & \lstinline|['A','B','A','B']| & 列表重复\\ \hline
\lstinline|3 in [1,2,3]| & \lstinline|True| & 成员检测\\ \hline
\lstinline|for x in [1,2,3]: print(x)| & \lstinline|1 2 3| & 迭代遍历\\ \hline
\end{tabular}
\end{block}
\end{frame}

\section{列表的函数}
\begin{frame}[fragile]{列表的三大核心函数}
\begin{columns}[T]
\column{0.4\textwidth}
\begin{block}{基础函数}
\begin{tabular}{|l|l|}
\hline
\textcolor{cuhksz2}{函数} & \textcolor{cuhksz2}{功能} \\ \hline
\lstinline|len()| & 获取元素个数 \\ \hline
\lstinline|max()| & 找最大值 \\ \hline
\lstinline|min()| & 找最小值 \\ \hline
\end{tabular}
\end{block}

\column{0.6\textwidth}
\begin{exampleblock}{实战应用}
\begin{lstlisting}[language=Python]
grade = [88,92,79,95]
print("人数:",len(grade))
print("最高分:",max(grade))
print("最低分:",min(grade))
\end{lstlisting}
\end{exampleblock}
\end{columns}

\begin{alertblock}{注意}
\begin{itemize}
\item 混合类型列表比较需谨慎
\item 空列表调用 $\texttt{max/min}$ 会报错
\end{itemize}
\end{alertblock}
\end{frame}

\begin{frame}[fragile]{综合挑战}
\begin{block}{挑战 $1$-密码生成器}
\begin{lstlisting}[language=Python]
p = ['A','2','#']
password = p * 2 + ['9','z']
print("".join(password))  # 输出应该是?
\end{lstlisting}
\end{block}

\begin{block}{挑战 $2$-成绩分析}
\begin{lstlisting}[language=Python]
grade = [85, 92, 78] + [90, 88]
print("平均:", sum(grade)/len(grade))
print("差距:", max(grade)-min(grade))
\end{lstlisting}
\end{block}
\end{frame}

\section{列表常用方法}

\begin{frame}[fragile]{列表的超级工具箱}

\begin{block}{$\texttt{append()}$ - 添加元素}
\begin{lstlisting}[language=Python]
shopping = []
shopping.append("牛奶")
shopping.append("面包")
# ['牛奶', '面包']
\end{lstlisting}
\end{block}

\begin{block}{$\texttt{count()}$ - 元素计数}
\begin{lstlisting}[language=Python]
grade = [85,90,85,88,90,85]
print(grade.count(85))  # 输出 3
\end{lstlisting}
\end{block}
\end{frame}

\begin{frame}[fragile]{列表的超级工具箱}

\begin{block}{$\texttt{index()}$ - 查找位置}
\begin{lstlisting}[language=Python]
fruit = ["苹果","香蕉","橙子"]
print(fruit.index("香蕉"))  # 输出 1
# 找不到会引发ValueError
\end{lstlisting}
\end{block}

\begin{block}{$\texttt{pop()}$ - 移除元素}
\begin{lstlisting}[language=Python]
do = ["作业","购物","洗衣"]
end = do.pop(1) 
# end="购物", do=["作业","洗衣"]
\end{lstlisting}
\end{block}

\end{frame}

\begin{frame}[fragile]{列表的超级工具箱}

\begin{block}{$\texttt{remove()}$ - 删除匹配项}
\begin{lstlisting}[language=Python]
animal = ["狗","猫","鸟","猫"]
animal.remove("猫")  
print (animal) # ["狗","鸟","猫"]
# 只删除第一个匹配项
\end{lstlisting}
\end{block}

\begin{block}{$\texttt{reverse()}$ - 反转列表}
\begin{lstlisting}[language=Python]
num = [1,2,3,4]
num.reverse()
print(num)  # [4,3,2,1]
\end{lstlisting}
\end{block}

\end{frame}

\begin{frame}[fragile]{列表的超级工具箱}

\begin{block}{$\texttt{sort()}$ - 列表排序}
\begin{lstlisting}[language=Python]
grade = [88,92,75,85]
grade.sort()  # 默认升序
print(grade)  # [75, 85, 88, 92]
grade.sort (reverse=True) # 降序
print(grade) # [92, 88, 85, 75]
\end{lstlisting}
\end{block}

\begin{block}{$\texttt{pop()}$ vs $\texttt{remove()}$}
\begin{tabular}{|l|l|l|}
\hline
\textcolor{cuhksz1}{方法} & \textcolor{cuhksz1}{参数} & \textcolor{cuhksz1}{返回值} \\ \hline
pop() & 索引(可选) & 被移除的元素 \\ \hline
remove() & 元素值 & None \\ \hline
\end{tabular}
\end{block}
\end{frame}


\end{document}